%     Created: Mon Apr 28 05:00 PM 2014 P
% Last Change: Mon Apr 28 05:00 PM 2014 P
%

\documentclass[12pt]{article}
\renewcommand{\familydefault}{\sfdefault} 
\usepackage{fullpage}
%
\begin{document}
\title{Useful Vim Tricks}
\author{Andrea Manzo}
\date{\today}
\maketitle

\begin{itemize}
    \item capital letter marks - can be used when editing multiple files
    \item tabs 
        \begin{itemize}
            \item :tabn to open a new tab
            \item gt to move between tabs 
            \item :tabedit to open a file in a new tab
            \item :tabsplit to open a tab with the same buffer
        \end{itemize}
    \item registers - specify register to yank stuff into; e.g. \textbf{``fyy} yanks a line into register f
    \item saveas - \textbf{:saveas }to save the file under a new name and edit the new file
    \item :split - splits the window; view the same file in two windows
        \begin{itemize}
            \item \textbf{CTRL-w w} switches between windows
            \item :only closes all windows except current one
            \item :vsplit opens windows side-by-side
            \item move window up: CTRL-W K
            \item move to the top window: CTRL-W k
            \item :qall closes all windows and exits vim
            \item :wall to write all windows
        \end{itemize}
    \item vimdiff to view differences between files
        \begin{itemize}
            \item \textbf{:diffsplit file} to view differences from within vim
        \end{itemize}
    \item recording mode:
        \begin{itemize}
            \item \textbf{q{register}} records keystrokes into register {register} < (a-z)
            \item \textbf{q} ends recording
            \item \textbf{@{register}} to execute macro
        \end{itemize}
        

\end{itemize}

\end{document}

