%     Created: Mon Apr 28 05:00 PM 2014 P
% Last Change: Mon Apr 28 05:00 PM 2014 P
%

\documentclass[12pt]{article}
\renewcommand{\familydefault}{\sfdefault} 
\usepackage{fullpage}
\usepackage{bookmark}

%
\begin{document}
\title{Useful Vim Tricks}
\author{Andrea Manzo}
\date{\today}
\maketitle

\section{Useful Plugins}
\begin{itemize}
    \item Latexbox - useful latex features
    \item snipmate - snippets for different filetypes
    \item tagbar - brings up a window that acts like a table of contents for
        various files 
    \item nerdcommenter - keybindings for commenting multiple filetypes
    \item vim-latex-live-preview - live preview of latex code
    \item CSApprox - allows vim to use fancy colorschemes
    \item pathogen - plugin manager
\end{itemize}

\section{My Vim configuration}
\begin{itemize}
    \item Have pathogen installed to manage plugins
    \item Filetype specific configurations are in \textasciitilde/.vim/ftplugin
    \item snippets for snipmate are in \textasciitilde/.vim/snippets
\end{itemize}

\section{Vim tricks for configuring}
\begin{itemize}
    \item :set filetype - shows what filetype options are set
    \item :set foldmethod - shows foldmethod
    \item :set rtp? - shows the directories in the run time path
    \item in .vimrc, call pathogen near the top of the file. Have the first
        lines be:
\begin{verbatim}
        set nocompatible
        call pathogen#infect()
\end{verbatim}
\end{itemize}

\section{Tricks from the Documentation}
\begin{itemize}
    \item capital letter marks - can be used when editing multiple files
    \item tabs 
        \begin{itemize}
            \item :tabn to open a new tab
            \item gt to move between tabs 
            \item :tabedit to open a file in a new tab
            \item :tabsplit to open a tab with the same buffer
        \end{itemize}
    \item registers - specify register to yank stuff into; e.g. \textbf{``fyy} yanks a line into register f
    \item saveas - \textbf{:saveas }to save the file under a new name and edit the new file
    \item :split - splits the window; view the same file in two windows
        \begin{itemize}
            \item \textbf{CTRL-w w} switches between windows
            \item :only closes all windows except current one
            \item :vsplit opens windows side-by-side
            \item move window up: CTRL-W K
            \item move to the top window: CTRL-W k
            \item :qall closes all windows and exits vim
            \item :wall to write all windows
        \end{itemize}
    \item vimdiff to view differences between files
        \begin{itemize}
            \item \textbf{:diffsplit file} to view differences from within vim
        \end{itemize}
    \item recording mode:
        \begin{itemize}
            \item \textbf{q{register}} records keystrokes into register {register} < (a-z)
            \item \textbf{q} ends recording
            \item \textbf{@{register}} to execute macro
        \end{itemize}
    \item substitute - \textbf{:s} to find/replace
        \begin{itemize}
            \item can write a command range (\% = apply to all lines)
            \item can use a pattern for a range, add and subtract lines, and use marks
            \item ``The command  ``I\{string\}Esc'' inserts the text {string} in each line'' 
        \end{itemize}
    \item ranges - after : but before command, can specify a range for the command to apply
        \begin{itemize}
            \item . = current line, \# = line \#s, \$ = last line
        \end{itemize}
    \item global - to execute more complicated commands
    \item visual block
        \begin{itemize}
            \item insert text at the end of a line: Visual block -> use \$ to extend to the end of the line, and then type ``A'' to append text
            \item ~ swap case, u to make lowercase, U uppercase
            \item \textgreater  to indent
        \end{itemize}
    \item \textbf{:read} to copy a file into vim, \textbf{:read | date} to read the output of a command into the file
    \item gq to format text
    \item use an external program: ex., !sort
    \item recover from a crash with the swap file
    \item "Clever Tricks" section: Count words, find/view a man page
\end{itemize}

\end{document}

